%% LyX 1.1 created this file.  For more info, see http://www.lyx.org/.
%% Do not edit unless you really know what you are doing.
\documentclass[american]{article}
\usepackage[latin1]{inputenc}
\usepackage{babel}

\makeatletter

%%%%%%%%%%%%%%%%%%%%%%%%%%%%%% LyX specific LaTeX commands.
\providecommand{\LyX}{L\kern-.1667em\lower.25em\hbox{Y}\kern-.125emX\@}

\makeatother
\begin{document}

\title{The RASTA Framework}


\author{Joel Becker}


\date{October 3, 2001}

\maketitle
\begin{abstract}
RASTA is an framework for describing tasks on a computer system. It
is well known that casual and non-expert users prefer to be guided
through tasks rather than trying to learn and understand a complex
and often esoteric system. This has lead to the common paradigms of
the {}``Wizard'' and the {}``Properties Dialog'' in current computer
software. A wizard takes the user step by step through a task, asking
pointed questions along the way. When the user has answered these
questions, the proper action is taken. This reduces the chance for
error, and allows the user to have a more satisfactory experience.
The property dialog is similar, though it often has merely one or
two steps, preferring to ask all the questions at once. Each approach
has its merits in certain contexts. The RASTA framework takes the
job of displaying wizards and dialogs out of the hands of the application
programmer. The designer of a RASTA setup creates a description of
the tasks and questions, and the RASTA system uses these descriptions
to guide the user through the tasks.
\end{abstract}

\section{Introduction}

Menu driven applications guide a user through a series of steps to
a final goal. This can take the form of a wizard, a dialog, or any
combination thereof. The user follows the guided path, and in the
end the user's desired action is completed. The better defined a task
is, the easier it is to handle all the appropriate questions required
of the user.

While specific software can be written for the task, writing software
for every task soon becomes a time-sink. A programmer has to spend
time defining these dialogs and wizards in order to expose the more
complex underpinnings of a system to the user. This takes development
time and effort. Diverse programming styles and conventions lead to
inconsistent wizards and dialogs, which can confuse the user, who
sees differing behavior on the same system. This is especially noticeable
in free software, where personal taste and whim are the rule and centralized
style convention is the exception.

The more general the application framework, the easier it is to modify
the selection of tasks, the questions, and the resulting actions.
RASTA is designed to be the end result of such an exercise. The library
and application know nothing about any task, but take descriptions
and render the choices and questions for the user. Maintenance of
the task information requires no change to the RASTA application and
can even be deployed long after the RASTA application has been installed
and configured.

RASTA is intended to fit any and all task that can be thought of.
In order to do this, a lot of thought has gone into the description
capabilities. It is a general solution, and so some questions may
not fit perfectly, but the intention is that all questions will be
possible. In this way, the system can be adapted to almost anything,
from system administration to data entry.


\section{A Generic Solution}

There are many reasons a generic solution makes sense. Ease of maintenance
has already been touched upon. The maintainer of the description files
does not need to know the C programming language, or any compiled
language at all. They merely update the description file to describe
their tasks.

The included front-end programs are intended to satisfy most requirements.
However, if a different front-end is required for some reason, the
application programmer can write one without having to worry about
the description file. When complete, it will read the description
file and behave identically to the included front-ends. This frees
the application programmer from worrying about the description methodology.
They merely have to follow the contract specified by the RASTA library
and fit their front-end to their specific need.

There is a known disadvantage to the generic approach. It is possible
that an interaction with the user can be done in a prettier or more
elegant manner when written with the specific interaction in mind.
A generic solution cannot predict these situations, and merely provides
facilities that should cover the situation in a slightly less optimal
manor. What matters is that the generic solution can still acquire
the required information.

The generic solution has advantages that outweigh this limitation
for many things. A user that is used to the interface of a RASTA application
already knows how to use and navigate the system. They can discover
new tasks and complete them with ease. A specific solution may have
its own style and interface, and may present a challenge to the user.
Multiple task domains can be coalesced into one RASTA description.
This gives the user a single starting point for all of the tasks.
The user merely navigates the task choices until they find the task
they want.


\section{The RASTA Architecture}

RASTA is intended to be of interest to a wide variety of administrators
and users. Different people have different philosophies on interaction
with a computer system. The new user wants their hand held, but the
experienced user often wants the quickest path to completion. RASTA
tries to provide solutions for all parts of this spectrum.

A user (of any experience level) interacts with RASTA via a front-end
application. This is an application that renders the choices and questions
that the description calls for, retrieves the user's responses, and
then runs the given task. It informs the user of the outcome, and
then is ready to start another task.

The initial RASTA implementation contains two front-ends. The first,
\texttt{clrasta}, is a command line implementation. It runs in a terminal
window and asks questions one by one. The other, \texttt{gtkrasta,}
is a GUI implementation using the GTK+ widget set. It is more complete,
and is currently used in real working environments.

The important thing here is that these front-ends share the same back-end
for reading and navigating the description. This back-end is a shared
library with a defined API. This allows other application programmers
to write front-ends utilizing this API. These new front-ends would
then be able to work with the description in a nearly identical fashion
to the current ones.


\section{The Description File}

The heart of RASTA's flexibility is the description file. The description
defines hierarchies of tasks that the user can navigate. Once the
user has selected a task, the described questions are asked, gathering
the information required to run the task. Finally, the task is executed.

The XML file format was chosen because the world is familiar with
the HTML and XML formats. XML has become a de facto standard for textual
interchange. The goal of this system is to get an installation up
quickly and easily, not to force others to learn another new and obscure
syntax. There is already a prototype editor for the description format,
but it is not ready for real use.


\subsection{Screens}

The fundamental unit of description is a screen. The RASTA application
displays a screen at a time. The user passes from screen to screen
until all the information required has been acquired. There are four
types of screens, three of which the user sees.

The first screen a user would likely see is a MENU screen. The MENU
screen presents a choice of tasks. The choices may be specific tasks,
or generic categories of tasks. The user selects the appropriate path,
and is taken to the next screen. If the choice was a generic category
of tasks, the user is presented with another MENU screen displaying
a more specific set of choices. The user follows these choices until
a specific task is selected.

If a specific task requires more information to continue, the user
is presented with a DIALOG screen. This screen asks the user one or
more questions pertaining to the task. The user answers, and then
proceeds to the next screen. There may be one DIALOG screen asking
all of the relevant questions, or there may be multiple DIALOG screens,
each asking a subset of the questions, ala a wizard.

The ACTION screen executes the task with the given information. The
\texttt{clrasta} and \texttt{gtkrasta} front-ends display the output
of the task, though this is a policy decision of the application programmer.
The screen then displays the outcome of the task -- success or failure
-- and then waits for the user to continue. The user can back up the
path to make another selection, redo this task, or exit.

The fourth screen, the HIDDEN screen, is unseen by the user and exists
to perform verification and validation tasks for the description.


\subsection{Paths}

Screens describe how each choice or question should be presented to
the user. The path describes the order in which screens are traveled.
The choices of a MENU screen are merely the possible paths beneath
it. The path description is a separate entity so that screens can
be reused along different paths. The description writer must make
sure that a screen used along multiple paths satisfies the needs of
each path. If this is the case, then maintenance of the paths becomes
easier, as a change to the screen definition changes all paths that
use the screen.


\subsection{External Scripts}

The descriptions require various interactions with the system. Fundamentally,
the description has to specify a task to execute. This is not the
only interaction required, however. Tasks can be run repeatedly. It
is required that the system have the ability to pre-fill fields with
already given answers. Consider a phone book system. The user wants
to update the phone number for their friend John. The DIALOG screen
appears, asking via a field what John's phone number is. If the system
already has an old phone number, the field should contain that number
instead of being blank. There needs to be a mechanism for the description
to do this. The mechanism needs to be generic as well.

All external interaction is done through shell scripts. This would
presumably map to the scripting capabilities of Windows on that platform.
The shell scripts in the description file allow substitution of already
known values from previous questions, and they may output text in
a specific format to pre-fill answers. This allows the most flexibility
for the designer of the descriptions. They may merely execute application
programs from these scripts, or they may have complicated script logic.
Between the XML and the scripting capabilities, this framework should
be able to handle almost any situation.


\section{Why RASTA}

When I originally conceived this application, it was in the context
of a system management tool. I didn't know what to call it, so I spent
some time thinking about it. YAST (Yet Another System Tool) was relatively
new at the time, so its ha-ha-only-serious name was already taken.
I could not think of anything sufficiently creative. I wanted a name
that \emph{wasn't} {}``System Tool'', but was still descriptive.
I also knew that creative acronyms got annoying after a while. Hence,
the {}``Really Annoying System Tool Acronym'', or RASTA.
\end{document}
